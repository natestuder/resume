%%%%%%%%%%%%%%%%%%%%%%%%%%%%%%%%%%%%%%%%%
% Friggeri Resume/CV
% XeLaTeX Template
% Version 1.2 (3/5/15)
%
% This template has been downloaded from:
% http://www.LaTeXTemplates.com
%
% Original author:
% Adrien Friggeri (adrien@friggeri.net)
% https://github.com/afriggeri/CV
%
% License:
% CC BY-NC-SA 3.0 (http://creativecommons.org/licenses/by-nc-sa/3.0/)
%
% Important notes:
% This template needs to be compiled with XeLaTeX and the bibliography, if used,
% needs to be compiled with biber rather than bibtex.
%
%%%%%%%%%%%%%%%%%%%%%%%%%%%%%%%%%%%%%%%%%

\documentclass[print]{template/friggeri-cv} % Add 'print' as an option into the square bracket to remove colors from this template for printing

%\addbibresource{bibliography.bib} % Specify the bibliography file to include publications

\begin{document}

\header{Nathan}{Studer}{Software Architect} % Your name and current job title/field

\begin{aside} % In the aside, each new line forces a line break
\section{Contact}
1051 Edna St. SE
Grand Rapids, MI 49507
USA
~
+1 (616) 648 4491
~
\href{mailto:nate.studer@gmail.com}{nate.studer@gmail.com}
%github?
%\href{http://www.smith.com}{http://www.smith.com}
%linkedin
%\href{http://facebook.com/johnsmith}{fb://jsmith}
\section{Languages}
English:  Native Language
German:  Elementary Proficiency (ILR Level 1)
\section{Programming Languages}
C, C++, C\#, Ada, VHDL, Verilog, python, PPC Assembly, ARM Assembly, Shell Scripting, Javascript, HTML5, CSS
\section{Operating Systems}
Unix, Linux, VxWorks, VxWorks 653, Windows, Xen
\section{Architectures}
PPC, ARM, FPGA SoC (Xilinx Zynq, Actel SmartFusion2)
\section{IDEs}
Microsoft Visual Studio, Eclipse, Android Studio, Freescale Codewarrior, TI CodeComposer, VxWorks Workbench, Xilinx ISE, Xilinx Vivado, Altera Quartus, Actel Libero.
\section{Revision Control}
Dimensions, CVS, git, SVN, mercurial.
\end{aside}

\section{Education}

\begin{entrylist}

\entry
{2006--2009}
{Masters {\normalfont of Science in Computer Science}}
{Michigan State University}
{Cumulative GPA:  3.887 (out of 4.000)}

\entry
{2001--2006}
{Bachelor {\normalfont of Science in Electrical Engineering (with Honors)}}
{Calvin College}
{\textbf{Bachelor} of Arts in German \\
Cumulative GPA:  3.824 (out of 4.000) \\
\\
\emph{Senior Design Project}: Programmed an autonomous go-cart built on the Real
Time Linux Kernel for entry in the International Ground Vehicle Competition (IGVC).}

\end{entrylist}

\section{Experience}

\begin{entrylist}

\entry
{2014--Now}
{Delphi}
{Auburn Hills, Michigan}
{\emph{Software Architect} \\
Software Architect in the North American Advanced Development group.  Worked with international team to respond to customer requests and develop and demonstrate technology to be used in future automobiles.  Was utilized as the group�s primary hypervisor and safety resource.
\begin{itemize}
\item Advanced Development
\begin{itemize}
\item Implemented HID device driver for Gaze and Gesture control of an HMI.  Also ported this work to an Arduino microcontroller.
\item HMI development.  HMI work was primarily HTML5 based with Javascript being used for dynamic content, which was later ported to Qt and QML.
\item Implemented safety fail-over PoC demo for CES 2015 using the Nvidia hypervisor to demonstrate how a hypervisor can provide system redundancy.
\item Managed suppliers of HMI assets and code for CES 2015 to successfully deliver a working demo HMI in time for CES.
\end{itemize}
\item Technology Proposals and Demonstrations
\begin{itemize}
\item Main demonstrator of Auburn Hills Advanced In-Vehicle Infotainment (IVI) group, creating and demoing technology demos at conferences and for customers.
\item Developed software and network architecture for several Rear Seat Entertainment (RSE) pursuit programs.
\end{itemize}
\item Next Generation Infotainment Platform (Integrated Cockpit) Development.
\begin{itemize}
\item Defined roadmap and software architecture for automotive hypervisor based systems.
\item Interfaced with several hypervisor vendors and evaluated their products.
\item Defined embedded software architecture for secure Over The Air (OTA) update.
\item Interfaced with several Over the Air vendors and evaluated their products.
\item Evaluated SoC platforms for next generation automotive use.
\item Developed and maintained BSPs based on x86 and ARM processors running Linux and Android respectively.
\end{itemize}
\end{itemize}}

\end{entrylist}

\newpage

\begin{aside2} % In the aside, each new line forces a line break
\section{Requirements Management}
Doors
\section{FPGA Simulation and Synthesis}
Mentor Graphics ModelSim, Mentor Graphics Precision, Synplify.
\section{Other}
WindChill, gcc, LaTeX, make, Microsoft Office.
\section{Protocols}
SPI, I2C, UART, LIN, CAN, MIL-STD-1553, AFDX, ARINC 429, TCP/IP, PCI
\section{Interfaces}
SDRAM, DRAM, FLASH, SRAM, LCD, GPS, BT656.
\section{Drivers}
SPI, I2C, EEPROM, FLASH, Ethernet, SMBus, PWM, HDLC, RS-232, RS-422, RS-485, PCI, DMA, VME.
\section{Professional Interests}
Hardware Emulation, Commercial Detection, Open Source Software/Hardwre, Functional Programming, Logic Programming, Artificial Intelligence
\section{Personal Interests}
Soccer, Weight-Lifting
\end{aside2}

\begin{entrylist}

\entry
{2006--2014}
{DornerWorks}
{Grand Rapids, Michigan}
{\emph{Electrical Engineer} \\
Served as an embedded engineer at an engineering services firm.  A highly rated and flexible team member with the ability to quickly acclimate to new tasks.  Utilized across all disciplines of embedded engineering.  
\\
\\
Completed master degree program while working full time at this position.
}

\entry
{2004--2006}
{Smiths Industries Aerospace}
{Grand Rapids, Michigan}
{\emph{Custom Logic Intern} \\
Tested and verified a custom logic hardware design for use as a component of commercial aerospace computer architecture.  This ASIC was designed to DO-254 level A standards.  
\\
\\
Completed undergraduate degree program while working part time at this position.
}

\end{entrylist}

\section{Patents}

\begin{entrylist}
\entry
{2014}
{\emph{System and Method for Deterministic Time Partitioning of Asynchronous Tasks in a Computing Environment.} (Provisional)}
{VanderLeest, Steven and Studer, Nathan.}
{Patent desribing a scheduling algorithm that maintains deterministic allocation of time with interrupts.}
\end{entrylist}

\section{Conference Presentations}

\begin{entrylist}
\entry
{2014}
{\emph{Xen and the Art of Certification}}
{Xen Developer Summit}
{Presentation describing software safety and security certification and how the Xen hypervisor could be certified.}
\end{entrylist}

\section{References}

\begin{entrylist}
\nodateentry
{Matt Remijn}
{Former Manager}
{+1 (616) 389-6141 \href{mailto:Matt.Remijn@dornerworks.com}{Matt.Remijn@dornerworks.com}}
\nodateentry
{Steve VanderLeest}
{Former Professor/Supervisor}
{+1 (616) 389-8315 \href{mailto:Steve.VanderLeest@dornerworks.com}{Steve.VandeerLeest@dornerworks.com}}
\nodateentry
{John Van Enk}
{Former Co-Worker}
{+1 (616) 826-7944 \href{mailto:vanenkj@gmail.com}{vanenkj@gmail.com}}
\end{entrylist}

%----------------------------------------------------------------------------------------
%	PUBLICATIONS SECTION
%----------------------------------------------------------------------------------------

%\section{publications}

%\printbibsection{article}{article in peer-reviewed journal} % Print all articles from the bibliography

%\printbibsection{book}{books} % Print all books from the bibliography

%\begin{refsection} % This is a custom heading for those references marked as "inproceedings" but not containing "keyword=france"
%\nocite{*}
%\printbibliography[sorting=chronological, type=inproceedings, title={international peer-reviewed conferences/proceedings}, notkeyword={france}, heading=bibheading]
%\end{refsection}

%\begin{refsection} % This is a custom heading for those references marked as "inproceedings" and containing "keyword=france"
%\nocite{*}
%\printbibliography[sorting=chronological, type=inproceedings, title={local peer-reviewed conferences/proceedings}, keyword={france}, heading=bibheading]
%\end{refsection}

%\printbibsection{misc}{other publications} % Print all miscellaneous entries from the bibliography

%\printbibsection{report}{research reports} % Print all research reports from the bibliography

%----------------------------------------------------------------------------------------

\end{document}
