%%%%%%%%%%%%%%%%%%%%%%%%%%%%%%%%%%%%%%%%%
% Friggeri Resume/CV
% XeLaTeX Template
% Version 1.2 (3/5/15)
%
% This template has been downloaded from:
% http://www.LaTeXTemplates.com
%
% Original author:
% Adrien Friggeri (adrien@friggeri.net)
% https://github.com/afriggeri/CV
%
% License:
% CC BY-NC-SA 3.0 (http://creativecommons.org/licenses/by-nc-sa/3.0/)
%
% Important notes:
% This template needs to be compiled with XeLaTeX and the bibliography, if used,
% needs to be compiled with biber rather than bibtex.
%
%%%%%%%%%%%%%%%%%%%%%%%%%%%%%%%%%%%%%%%%%

\documentclass[print]{template/friggeri-cv} % Add 'print' as an option into the square bracket to remove colors from this template for printing

%\addbibresource{bibliography.bib} % Specify the bibliography file to include publications

\begin{document}

\header{Nathan}{Studer}{Senior Engineer} % Your name and current job title/field

\begin{aside} % In the aside, each new line forces a line break
\section{Contact}
13445 Red Leaf Court
Nunica, MI 49448
USA
~
+1 (616) 648 4491
~
\href{mailto:nate.studer@gmail.com}{nate.studer@gmail.com}
%github?
%\href{http://www.smith.com}{http://www.smith.com}
%linkedin
%\href{http://facebook.com/johnsmith}{fb://jsmith}
\section{Languages}
English:  Native Language
German:  Elementary Proficiency (ILR Level 1)
\section{Programming Languages}
C, C++, C\#, Ada, VHDL, Verilog, python, PPC Assembly, ARM Assembly, Shell Scripting, Javascript, HTML5, CSS
\section{Operating Systems}
Unix, Linux, VxWorks, VxWorks 653, Windows, Xen
\section{Architectures}
PPC, ARM, FPGA SoC (Xilinx Zynq, Actel SmartFusion2)
\section{IDEs}
Microsoft Visual Studio, Eclipse, Android Studio, Freescale Codewarrior, TI CodeComposer, VxWorks Workbench, Xilinx ISE, Xilinx Vivado, Altera Quartus, Actel Libero.
\section{Revision Control}
Dimensions, CVS, git, SVN, mercurial.
\end{aside}

\section{Education}

\begin{entrylist}

\entry
{2006--2009}
{Masters {\normalfont of Science in Computer Science}}
{Michigan State University}
{Cumulative GPA:  3.887 (out of 4.000)}

\entry
{2001--2006}
{Bachelor {\normalfont of Science in Electrical Engineering (with Honors)}}
{Calvin College}
{\textbf{Bachelor} of Arts in German \\
Cumulative GPA:  3.824 (out of 4.000) \\
\\
\emph{Senior Design Project}: Programmed an autonomous go-cart built on the Real
Time Linux Kernel for entry in the International Ground Vehicle Competition (IGVC).}

\end{entrylist}

\section{Experience}

\begin{entrylist}

\entry
{2016--Now}
{DornerWorks}
{Grand Rapids, Michigan}
{\emph{Senior Engineer} \\
Senior Engineer at an engineering services firm with a primary focus on hypervisor and FPGA based systems.
\begin{itemize}
\item Obtained Secret clearance.
\item Developed cooperative paravirtual driver for Xilinx AXI DMA Engine.
\item Ported Xen to i.MX8.
\item Developed and presented hypervisor training and marketing material.
\item Architected and implemented the software architecture for a seL4 based augmented reality helmet system.
\item Converted internal Linux build system to Yocto.
\item Maintained Xen based distribution for Xilinx's Zynq UltraScale and NXP's i.MX8.
\end{itemize}}

\end{entrylist}

\begin{entrylist}

\entry
{2014--2016}
{Delphi}
{Auburn Hills, Michigan}
{\emph{Software Architect} \\
Software Architect in the North American Advanced Development group.  Worked with international team to respond to customer requests and develop and demonstrate technology to be used in future automobiles.  Was utilized as the group’s primary hypervisor and safety resource.
\begin{itemize}
\item Advanced Development
\begin{itemize}
\item Implemented HID device driver for Gaze and Gesture control of an HMI.  Also ported this work to an Arduino microcontroller.
\item HMI development.  HMI work was primarily HTML5 based with Javascript being used for dynamic content, which was later ported to Qt and QML.
\item Implemented safety fail-over PoC demo for CES 2015 using the Nvidia hypervisor to demonstrate how a hypervisor can provide system redundancy.
\item Managed suppliers of HMI assets and code for CES 2015 to successfully deliver a working demo HMI in time for CES.
\end{itemize}
\item Technology Proposals and Demonstrations
\begin{itemize}
\item Main demonstrator of Auburn Hills Advanced In-Vehicle Infotainment (IVI) group, creating and demoing technology demos at conferences and for customers.
\item Developed software and network architecture for several Rear Seat Entertainment (RSE) pursuit programs.
\end{itemize}
\item Next Generation Infotainment Platform (Integrated Cockpit) Development.
\begin{itemize}
\item Defined roadmap and software architecture for automotive hypervisor based systems.
\item Interfaced with several hypervisor vendors and evaluated their products.
\item Defined embedded software architecture for secure Over The Air (OTA) update.
\item Interfaced with several Over the Air vendors and evaluated their products.
\item Evaluated SoC platforms for next generation automotive use.
\item Developed and maintained BSPs based on x86 and ARM processors running Linux and Android respectively.
\end{itemize}
\end{itemize}}

\end{entrylist}

\newpage

\begin{aside2} % In the aside, each new line forces a line break
\section{Linux Systems}
Ubuntu, Debian, buildroot, Yocto
\section{Requirements Management}
Doors
\section{FPGA Simulation and Synthesis}
Mentor Graphics ModelSim, Mentor Graphics Precision, Synplify.
\section{Other}
WindChill, gcc, LaTeX, make, Microsoft Office.
\section{Protocols}
SPI, I2C, UART, LIN, CAN, MIL-STD-1553, AFDX, ARINC 429, TCP/IP, PCI
\section{Interfaces}
SDRAM, DRAM, FLASH, SRAM, LCD, GPS, BT656.
\section{Drivers}
SPI, I2C, EEPROM, FLASH, Ethernet, SMBus, PWM, HDLC, RS-232, RS-422, RS-485, PCI, DMA, VME.
\section{Professional Interests}
Hardware Emulation, Commercial Detection, Open Source Software/Hardwre, Functional Programming, Logic Programming, Artificial Intelligence
\section{Personal Interests}
Soccer, Weight-Lifting
\end{aside2}

\begin{entrylist}

\entry
{2006--2014}
{DornerWorks}
{Grand Rapids, Michigan}
{\emph{Electrical Engineer} \\
Served as an embedded engineer at an engineering services firm.  A highly rated and flexible team member with the ability to quickly acclimate to new tasks.  Utilized across all disciplines of embedded engineering.  
\\
\\
Completed master degree program while working full time at this position.
\begin{itemize}
\item Technical Lead on a project to develop an DO-178B/C certified Open Source ARINC653 Operating System using Linux and Xen.
\begin{itemize}
\item Co-Maintained the arinc653 real-time scheduler in Xen.
\item Developed InterPartition Communication (IPC) Software for use in a Linux Operating System running on top of the Xen hypervisor.
\item Implemented ARINC653 RS232 and CAN IPC drivers.
\item Ported and demoed Xen on a military radio platform based on an ARM processor without virtualization extensions.
\item Wrote the DO-178C planning documents for the project using an innovative Agile/Waterfall hybrid process framework, which utilized Agile methods to increase the frequency of Stage Of Involvement (SOI) audits and avoid costly rework in late stages of certification.
\item Extended RMTOO to create a custom requirements management solution.
\item Developed DO-178C requirement and design artifacts for the Interpartition Communication Software and the Event Channel feature of Xen.
\item Classified data objects/types in order to perform a semi-formal data flow security analysis of Xen.
\end{itemize}
\item Safety Critical Software Engineering
\begin{itemize}
\item Developed software drivers and application interfaces for SPI, I2C, UART, SMBus, EEPROM, and FLASH devices for a smart battery application used in a medical setting.
\item Emulated the electrical behavior of an aircraft fan assembly using a PIC microprocessor.
\item Requirements management and requirements traceability as well as architecture definition and design for the VxWorks 653 board support package of a custom commercial aerospace CPU architecture in use on the Boeing 787.  This development was certified to DO-178B level A standards.  The system included multiple hardware and software redundancies.
\item Resolved PRs for the BSP and AFDX drivers of a commercial aerospace OS.
\item Performed object code analysis, to provide DO-178B evidence of source to object traceability.
\item Developed and ran requirements based testing for the AFDX, BSP, HMON, and FileSystem components of a commercial aerospace OS.
\item Developed and ran robustness testing software for a Jettison and Interlock military aerospace system.
\item Created drivers and board support package for several custom aviation computer architectures based on the VxWorks and VxWorks 653 operating system.  Drivers included DMA, VME, PCI, SPI, Ethernet, RS-232/422/485, and HDLC.
\item Ported several VxWorks board support packages to an equivalent VxWorks 653 board support packages.
\item Developed the software for a custom DSP board, which was used to verify the operation of the National Instrument test harness.  The software was developed separately from the production software to meet DO-178B’s independence criteria.
\item Developed a test application to verify the functionality of a TCP/IP stack to relevant Request For Comments (RFCs).
\end{itemize}
\end{itemize}}

\end{entrylist}

\let\oldclearpage\clearpage
\renewcommand{\clearpage}{}
\newgeometry{left=1.5cm,top=1.5cm}
\let\clearpage\oldclearpage

\begin{entrylist}

\listentry
{\begin{itemize}
\item Custom Logic
\begin{itemize}
\item Technical Lead on custom logic reuse effort.
\item Designed and Implemented an FPGA based system used as a Data Crash Recorder for capturing CAN/LIN traffic during vehicle crash testing.
\item Designed and implemented an FPGA SoC which provided control of an Ultrasonic Aspirator.  Both hardware and software of the SoC were used to drive the physical element and keep it at resonance.  Through creative use and cooperation of the SoC’s processor and FPGA fabric the BOM cost of the control system was greatly reduced.
\item Developed the testbench, bus functional models, and/or testcases for several military FPGAs.
\item Designed and implemented a television video processing FPGA.
\begin{itemize}
\item Added crawllines and other graphic data to a live video stream while only incurring an 11 pixel delay.
\item Other functionality included input image capture, station identifier inseration, as well as fade in/out functionality.
\item All video processing was done within the FPGA with a web based control interface handled by a connected microprocessor.
\item Created the testbench, bus functional models, and testcases for the verification of this device.
\item Implemented PSRAM, SRAM, SDRAM, DRAM, BT656, and Serial FLASH interfaces.
\end{itemize}
\item Obsolescence Updates
\begin{itemize}
\item Developed a custom PCI target, to replace and emulate several obsolete hardware components on a custom PCI-CAN board used in a ventilator in a medical setting.
\item Update of several FPGAs in various legacy military applications including a voice and data recorder (ARINC TBD) and an inertial reference system (GPS, Gyro Caging Control System).
\item Update of several FPGAs in a fuel quantity measurement system in use on the Boeing 777.  This update was certified to DO-254 level A standards and was completed 50\% under budget.  (ARINC TBD)
\end{itemize}
\end{itemize}
\item Hardware Engineering
\begin{itemize}
\item Designed and performed the layout for a custom IrDa communication board for use in a hospital setting.
\item Developed hardware requirements and traceability for the hardware components of a Jettison and Interlock military aerospace system.
\end{itemize}
\item System Engineering
\begin{itemize}
\item Designed and authored the technical solution for several commercial and safety critical project proposals.
\end{itemize}
\end{itemize}}

\end{entrylist}

\begin{entrylist}

\longentry
{2004--2006}
{Smiths Industries Aerospace}
{Grand Rapids, Michigan}
{\emph{Custom Logic Intern} \\
Tested and verified a custom logic hardware design for use as a component of commercial aerospace computer architecture.  This ASIC was designed to DO-254 level A standards.  
\\
\\
Completed undergraduate degree program while working part time at this position.
\begin{itemize}
\item Safety Critical Custom Logic Design and Testing.  Primary areas of focus included PCI, DMA, and the RS-232 interface.
\begin{itemize}
\item Implemented behavioral simulation modules of external devices.
\item Independently verified behavorial simulation modules created by others.
\item Created and ran manual and automated testcases to acheive 100\% code, branch, and Modified Condition/Decision (MC/DC) coverage.
\item Peer reviewed design and implementation of modules.
\item Traced requirements, design, and test implementation/results using DOORS.
\end{itemize}
\end{itemize}}

\end{entrylist}

\newpage

\section{Patents}

\begin{entrylist}
\longentry
{2014}
{\emph{System and Method for Deterministic Time Partitioning of Asynchronous Tasks in a Computing Environment.} (Provisional)}
{VanderLeest, Steven and Studer, Nathan.}
{Patent desribing a scheduling algorithm that maintains deterministic allocation of time with interrupts.}
\end{entrylist}

\section{Conference Presentations}

\begin{entrylist}
\longentry
{2014}
{\emph{Xen and the Art of Certification}}
{Xen Developer Summit}
{Presentation describing software safety and security certification and how the Xen hypervisor could be certified.}
\end{entrylist}

\section{References}

\begin{entrylist}
\nodateentry
{Jonathan Van Stensel}
{Former Co-Worker}
{+1 (616) 929-0064 \href{mailto:Jonathan.Van.Stensel@ge.com}{Jonathan.Van.Stensel@ge.com}}
\nodateentry
{Steve VanderLeest}
{Former Professor/Supervisor}
{+1 (616) 389-8315 \href{mailto:Steve.VanderLeest@dornerworks.com}{Steve.VandeerLeest@dornerworks.com}}
\nodateentry
{John Van Enk}
{Former Co-Worker}
{+1 (616) 826-7944 \href{mailto:vanenkj@gmail.com}{vanenkj@gmail.com}}
\end{entrylist}

%----------------------------------------------------------------------------------------
%	PUBLICATIONS SECTION
%----------------------------------------------------------------------------------------

%\section{publications}

%\printbibsection{article}{article in peer-reviewed journal} % Print all articles from the bibliography

%\printbibsection{book}{books} % Print all books from the bibliography

%\begin{refsection} % This is a custom heading for those references marked as "inproceedings" but not containing "keyword=france"
%\nocite{*}
%\printbibliography[sorting=chronological, type=inproceedings, title={international peer-reviewed conferences/proceedings}, notkeyword={france}, heading=bibheading]
%\end{refsection}

%\begin{refsection} % This is a custom heading for those references marked as "inproceedings" and containing "keyword=france"
%\nocite{*}
%\printbibliography[sorting=chronological, type=inproceedings, title={local peer-reviewed conferences/proceedings}, keyword={france}, heading=bibheading]
%\end{refsection}

%\printbibsection{misc}{other publications} % Print all miscellaneous entries from the bibliography

%\printbibsection{report}{research reports} % Print all research reports from the bibliography

%----------------------------------------------------------------------------------------

\end{document}
